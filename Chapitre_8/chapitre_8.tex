\chapter{Conclusion and perspectives}
\quad\ This chapter is here to summarize the content of the thesis. The purpose of the work was to provide some improvement to the existing Python computer code modeling a Brayton gas cycle. To understand how these improvements have been implemented, some theoretical contents were required to be introduced. 

The theory dispensed has been split into 3 chapters. In the chapters \ref{C2} and \ref{C3}, the basis thermodynamic has been presented. One of the points covered by the third chapter was about the assessment of the thermodynamic state of a fluid based on the knowledge of two independent state variables. The special case of the ideal gas has been emphasized.

Then, the chapter \ref{C4} focused on the theoretical contents about the components constituting the Brayton cycle. In particular, the principle of similarity for the turbomachines was introduced. The performance maps of the compressor and the turbine have been derived based on this principle. 

In the chapter \ref{C5}, a non-exhaustive list of variants of the Brayton gas cycle was proposed. Three of these variants have been analyzed. The purpose was to observe the impact of a regenerator and a bi-staged compression and expansion on the performance of the cycle. 

The last two chapters covered the work done during this master thesis. The object-oriented programming has been explored. This paradigm of programming has been found to be really powerful thanks to the flexibility of the computer code. The chapter \ref{C6} provided an overview of how the program is structured. 

Finally, the implementation of the different model of each component has been initiated. The main focus of the work was about the implementation of the turbomachines performance maps in the program. It has been found that the method to implement the map required some pre-processing of the data in order to provide an efficient polynomial regression. For instance, the multiplication of the pressure ratio by the rotational speed when building the relations shows that it \textit{smooths}
the polynomials generated by the least square regression algorithm.


The interpolation at the lower rotational speeds for the compressor was the most challenging part. This is due to the relative proximity of the operating points characterized by the same rotational speed. 

In particular, the variation of the pressure ratio $\Pi$ with respect to the other operating variables is almost nonexistent. This implies that the relations of the type $x(\Pi,N)$ are characterized by quasi-vertical iso-rotational speed curves. This problem has been solved by considering the inverse relations $\Pi(x,N)$ and performing an iterative search of the value of $x$.


Nevertheless, there are still areas that haven't been covered in this work. The principal one is the consideration of the different transient effects. 

To take into account the transient effects, it would be required to perform a deeper characterization of the different components of the Brayton cycle. For instance, the inertia of the assembly turbine+compressor+generator has been considered. The modeling of the flow within the system also needs to be studied.

Another field that hasn't been covered in this master thesis is the modeling of the generator used to convert the mechanical power generated by the turbine into electricity. Since it is a rotating machine, a characteristic map should be built in order to have relation between the generation of electricity and the rotational speed of the generator shaft.


Initially, one part of the work consisted in comparing the result from the computer code with ones that were obtained by the means of experimental campaigns. However, due to Covid-19 crisis and some issues regarding to the test bench, this part has been canceled due to the lack of data to be post-processed. Thus, the work initiated during this master thesis will be continued to provide new features continuously to the program. 

A graphical interface to interact with the program will be implemented. While this is possible to interact through the command line, the implementation of a graphical interface would allow easing the usage of the program.


