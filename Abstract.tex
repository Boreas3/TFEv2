\chapter*{Abstract}
\quad\ Since the last decades, there have been many breakthroughs in the development of thermodynamic cycles. This is thanks to the development of the different technologies used during the design and manufacturing phases of the different components (pumps, compressors, turbines, combustion chamber, etc.). By improving their efficiency, it has been possible to increase the compactness of the system. In particular, micro gas turbine based on the Brayton cycle started to be developed. 

The goal of this work is to provide a program written in Python that will assess the performance of different configurations of the Brayton gas cycle. Among the possible configurations, there are the Gas Turbine (GT), Regenerative Gas Turbine (RGT) and the Intercooler-Regenerative-Reheat Gas Turbine (IRHGT).  The program aims to utilize performance maps of the compressor and the   turbine to model a Brayton cycle for which the performance will vary with the rotational speed and the mass flow rate. Other developments on the evaluation of the pressure drops, efficiency of the heat exchangers based on nominal conditions, etc. are also considered.

To develop the program, theoretical contents of thermodynamics are first dispensed in order to understand the concept used in the implementation phase. Once the theory given, a general structure of the program is provided. Then a description of the gas turbine elements' model is given along with the numerical methods involved. 

In particular, the implementation of the performance maps using the least square regression (LS-R) algorithm is covered. The algorithm is applied on a database of operating points defined by the set of parameters $(\dot{m}_{corr},\Pi_{tt},N_{corr},\eta_{is})$. The purpose is to build relations of the type $f(x,y)$ that allow to fully describe the compressor and the turbine. The choice of these relations with the associated results of the regressions is discussed in the last chapter of this work.
